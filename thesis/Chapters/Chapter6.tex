% Chapter 1

\chapter{Conclusions and Future Directions} % Write in your own chapter title
\label{Chapter6}
\lhead{Chapter 1. \emph{Thesis Structure}} % Write in your own chapter title to set the page header

\section{Conclusion}
This project carried out the implementation of H.264 video encoder with modules for prediction, transformation, quantization, inverse transformation and encoding.  Significant enhancements were made to the Baseline Profile through the introduction of advanced frame prediction algorithms, resulting in the development of the Main Profile. The inter-prediction module developed for this project employs a 256 PE VBS ME hardware architecture. Extensive testing, including assessment of inter-prediction module, was conducted. The H.264 encoder was analyzed and synthesized using Vivado for Xilinx Artix-7, and was successfully integrated on FPGA, showing improved compression efficiency and potential for hardware acceleration in real-time video encoding. This thesis has provided a thorough analysis of the creation of a video encoder for an open-source camera chip for reliable and secure camera systems. In order to completely control the camera system, other teams from 10X and NUST worked to develop the camera from the ground up, putting any worries about data storage and security to rest. Scalar, vector, and CAVLC encoding techniques have been combined to effectively compress and encode camera data while keeping high-quality output for further processing. The project contributes to advanced H.264 encoding and provides valuable insights for future video compression research.

\section{Future Directions}
Our project is a part of a long term project that aims to develop an open source camera chip. The idea is to develop a main RISC-V based processor, a video processor, ISP and integrate them on ChipYard to fabricate an IC. Our open source camera chip may gain significant traction and adoption as the demand for sustainable and secure camera solutions grows in the market.

With evolving security concerns, the emphasis on improving the security features of camera systems with increase and our projects data protection and security will surely meet these requirements.

As, this project is open source it has a potential for customization and flexibility, this will allow engineers to tailor this chip's functionality to their specific needs.

New emerging technologies such as Artificial intelligence, edge computing and Machine learning can enhance the capability of camera systems by enabling features such as advanced image recognition, intelligent decision making and real-time analytic.

This project, after gaining enough credibility, may attract a community of developers and contributors who can contribute to enhance the capabilities of this chip.